\documentclass[12pt]{article}
\usepackage[utf8]{inputenc} % Asegura que el archivo esté en UTF-8
\usepackage[spanish]{babel} % Soporte para idioma español
\usepackage{graphicx}
\usepackage{hyperref}
\usepackage{amsmath}
\usepackage{listings}
\usepackage{xcolor}

% Configuración para código Python
\definecolor{codegreen}{rgb}{0,0.6,0}
\definecolor{codegray}{rgb}{0.5,0.5,0.5}
\definecolor{codepurple}{rgb}{0.58,0,0.82}
\definecolor{backcolour}{rgb}{0.95,0.95,0.92}

\lstdefinestyle{mystyle}{
    backgroundcolor=\color{backcolour},
    commentstyle=\color{codegreen},
    keywordstyle=\color{magenta},
    numberstyle=\tiny\color{codegray},
    stringstyle=\color{codepurple},
    basicstyle=\ttfamily\footnotesize,
    breakatwhitespace=false,
    breaklines=true,
    captionpos=b,
    keepspaces=true,
    numbers=left,
    numbersep=5pt,
    showspaces=false,
    showstringspaces=false,
    showtabs=false,
    tabsize=2
}

\lstset{style=mystyle}

% Título y autor
\title{\textbf{Guía Completa: Análisis de Datos con Python y Estadística Básica}}
\author{Tu Nombre}
\date{\today}

\begin{document}

\maketitle

\tableofcontents

\section{Introducción}
\subsection{¿Qué es un CSV?}
Un archivo CSV (Comma-Separated Values) almacena datos en forma de tabla, donde cada fila es un registro y cada columna está separada por comas (o puntos y comas, como en tu caso).

\subsection{¿Por qué usar Python?}
Python es un lenguaje de programación fácil de aprender con herramientas poderosas para análisis de datos, como las bibliotecas \texttt{pandas}, \texttt{numpy}, y \texttt{matplotlib}.

\section{Configuración Inicial}
\subsection{Instalar Bibliotecas}
\begin{lstlisting}[language=Python]
!pip install pandas numpy matplotlib seaborn
\end{lstlisting}
- **¿Qué hace?**: Instala las herramientas necesarias para trabajar con datos y gráficos.

\subsection{Importar Bibliotecas}
\begin{lstlisting}[language=Python]
import pandas as pd
import numpy as np
import matplotlib.pyplot as plt
import seaborn as sns
\end{lstlisting}
- **Lógica**: Cargamos las herramientas para usarlas con nombres más cortos (ej: \texttt{pd} en lugar de \texttt{pandas}).

\section{Cargar y Explorar Datos}
\subsection{Leer el CSV}
\begin{lstlisting}[language=Python]
df = pd.read_csv("prueba.csv", sep=";", encoding="utf-8")
\end{lstlisting}
- **¿Qué hace?**: Carga el archivo CSV en una tabla llamada \texttt{DataFrame} (\texttt{df}).

\subsection{Primeras Filas}
\begin{lstlisting}[language=Python]
df.head()
\end{lstlisting}
- **Salida**: Muestra las primeras 5 filas del DataFrame.

\subsection{Información General}
\begin{lstlisting}[language=Python]
df.info()
\end{lstlisting}
- **Salida**: Número de filas y columnas, nombres de columnas y tipos de datos.

\section{Limpieza de Datos}
\subsection{Valores Faltantes}
\begin{lstlisting}[language=Python]
df.isnull().sum()
\end{lstlisting}
- **¿Qué hace?**: Cuenta cuántos valores faltantes (\texttt{NaN}) hay en cada columna.

\subsection{Corregir Tipos de Datos}
\begin{lstlisting}[language=Python]
df["A\~no"] = df["A\~no"].astype(int) % Escapamos la "n~" con \textbackslash{}~n
\end{lstlisting}
- **¿Qué hace?**: Convierte la columna "Año" a números enteros.

\section{Análisis Estadístico}
\subsection{Matriz de Correlación}
\begin{lstlisting}[language=Python]
ventas = ["Ventas NA", "Ventas EU", "Ventas JP", "Ventas Otros", "Ventas Global"]
correlacion = df[ventas].corr()
sns.heatmap(correlacion, annot=True, cmap="coolwarm")
plt.show()
\end{lstlisting}
- **Interpretación**: Muestra cómo se relacionan las ventas entre regiones.

\section{Visualizaciones Clave}
\subsection{Gráfico de Barras (Top Plataformas)}
\begin{lstlisting}[language=Python]
plataformas = df["Plataforma"].value_counts().head(10)
plataformas.plot(kind="bar", title="Top 10 Plataformas")
plt.xlabel("Plataforma")
plt.ylabel("Cantidad de Juegos")
plt.show()
\end{lstlisting}
- **Lógica**: Cuenta cuántos juegos hay por plataforma y muestra las 10 más comunes.

\section{Modelos Estadísticos Básicos}
\subsection{Media y Mediana}
\begin{lstlisting}[language=Python]
df["Ventas Global"].mean()  % Media
df["Ventas Global"].median()  % Mediana
\end{lstlisting}
- **Media**: Promedio de ventas.
- **Mediana**: Valor que separa la mitad superior e inferior de los datos.

\section{Exportar Resultados}
\begin{lstlisting}[language=Python]
df.to_csv("datos_procesados.csv", index=False)
\end{lstlisting}
- **¿Qué hace?**: Guarda el DataFrame limpio y procesado en un nuevo CSV.

\section{Glosario}
\begin{itemize}
    \item \textbf{DataFrame}: Tabla de datos en pandas (filas y columnas).
    \item \textbf{NaN}: Valor faltante o nulo.
    \item \textbf{Correlación}: Relación estadística entre dos variables.
    \item \textbf{Outlier}: Dato anormalmente alejado de los demás.
\end{itemize}

\section{Consejos Finales}
\begin{itemize}
    \item Si un gráfico no se muestra, añade \texttt{\%matplotlib inline} al inicio del notebook.
    \item Usa \texttt{df.sample(5)} para ver filas aleatorias.
    \item Experimenta cambiando parámetros (ej: \texttt{kind="bar"} a \texttt{kind="pie"}).
\end{itemize}

\end{document}